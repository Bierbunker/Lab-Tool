\documentclass[11pt]{scrartcl}

% standard packages
\usepackage[utf8]{inputenc}  % input in UTF-8
\usepackage[T1]{fontenc}  % output in T1 fonts (westeuropäische Codierung)
\usepackage{lmodern}  % latin modern fonts
\usepackage[ngerman]{babel}  % deutsches Sprachpaket, neue Rechtschreibung

% Seitensetup
\usepackage{scrlayer-scrpage}  % Seitenformatierung durch KOMA-interne Optionen
\usepackage[top=4cm, bottom=4cm]{geometry}  % Seitengeometrie (kann durch KOMA ersetzt werden, hab ich aber nicht geschafft)
\usepackage[hypcap=false]{caption, subcaption}  % caption editing - hypcap warning with hyperref
\usepackage{array}  % table editing

% additional packages
\usepackage{amsmath, amssymb, amstext}  % math packages (American Math Society)
\usepackage{icomma}  % Kommata in Dezimalzahlen verursachen keinen Abstand mehr
\usepackage{graphicx}  % Bilder einfügen
\usepackage{pdfpages}  % PDF als vollständige Seiten einfügen
\usepackage{lastpage}  % referenziert die letzte Seite
\usepackage[separate-uncertainty=true]{siunitx}  % bessere Darstellung von Einheiten
%\usepackage{datatool}
\usepackage[hidelinks]{hyperref}  % hyperref verlinkt Referenzen - hidelinks entfernt borders um links

% package setups
% Kopf- und Fußzeile durch KOMA
\pagestyle{scrheadings}  % KOMA darf entscheiden
\clearpairofpagestyles  % reset
\setkomafont{pageheadfoot}{\normalfont}  % Standardschrift in Kopf- und Fußzeile
%\setlength{\headheight}{27.2pt}  % benötigte Höhe Kopfzeile (warning von scrlayer-scrpage, wird aber automatisch so gerendert, falls diese Option weggelassen wird)
\ihead{TITEL HIER EINFÜGEN}  % Kopf links %Todo Titel ändern
\chead{\textsc{Timberman} Markus}  % Kopf Mitte %Todo Name ändern
\ohead{31.02.2000}  % Kopf rechts %Todo Datum ändern
\cfoot{\pagemark \, / \pageref{LastPage}}  % Fuß Mitte

% Table of Contents
\DeclareTOCStyleEntry{dottedtocline}{section}  % KOMA intern - Inhaltsverzeichnis mit Punkten (nur sections)

% SI
\sisetup{locale = DE}  % deutschsprachige SI-Konvention

% array
\renewcommand{\arraystretch}{1.2}

\begin{document}

%\includepdf{Deckblatt_print.pdf} % Todo Deckblatt ausfüllen

\tableofcontents
\newpage

\section{Aufgabenstellung}
\label{sec:aufgabenstellung}

\section{Voraussetzungen und Grundlagen}
\label{sec:voraussetzungen_grundlagen}
Die folgende Gleichung \ref{eq:bewegungsgl} hab i wie alles andere auch von \cite[S. 69]{ref:Beispiel} geklaut.
\begin{equation}
    \label{eq:bewegungsgl}
        m \cdot g \cdot \sin(\varphi) = -m \cdot l \cdot \ddot \varphi 
\end{equation}

\section{Versuchsanordnung}
\label{sec:versuchsanordnung}

% \begin{figure}[htbp] %Todo add graphic
%     \centering
%     \caption[Versuchsanordnung]{Bla Bla}
%     \label{fig:Versuchsanordnung}  % after caption for right numbering
    
%     \includegraphics[width=0.35\textwidth]{NAME.png}
% \end{figure}

\section{Geräteliste}
\label{sec:geraeteliste}
\begin{center}
    \captionof{table}[Geräteliste]{Verwendete Geräte}  % optionales Argument wird in Verzeichnissen verwendet, essentielles Argument direkt im Text
    \label{tab:geraeteliste}
    \vspace{3mm}  % vertical space 3 mm
    \begin{tabular}{|c|c|c|c|}
        \hline
        Gerät           & Hersteller                & Modell        & Genauigkeitsklasse    \\ \hline
        Der Gerät    & Knoll GmbH    & Studentenzerstörer & $\infty$  \\ \hline
        Der Gerät    & Knoll GmbH    & Studentenzerstörer & $\infty$  \\ \hline
        \hline
    \end{tabular}
\end{center}

\section{Versuchsdurchführung und Messergebnisse}
\label{sec:versuchsdurchfuehrung_messergebnisse}
% \begin{center}
%     \captionof{table}[MESSUNG]{Bla Bla}%Ausfüllen
%     \label{tab:messergebnisse}

%     \vspace{3mm}
%     \begin{tabular}{|cc|cc|cc|}
%         \hline
%         n   & $10 \cdot T$  & n  & $10 \cdot T$ & n  & $10 \cdot T$ \\ \hline
%         1   & 29,12         & 11 & 29,11        & 21 & 29,06        \\ \hline
%         2   & 29,06         & 12 & 29,03        & 22 & 29,09        \\ \hline
%         3   & 29,16         & 13 & 29,06        & 23 & 29,06        \\ \hline
%         4   & 29,04         & 14 & 29,13        & 24 & 29,11        \\ \hline
%         5   & 29,04         & 15 & 29,03        & 25 & 29,08        \\ \hline
%         6   & 29,19         & 16 & 29,07        & 26 & 29,04        \\ \hline
%         7   & 29,13         & 17 & 29,06        & 27 & 29,04        \\ \hline
%         8   & 29,03         & 18 & 29,08        & 28 & 29,04        \\ \hline
%         9   & 29,10         & 19 & 29,06        & 29 & 29,03        \\ \hline
%         10  & 29,08         & 20 & 29,00        & 30 & 29,03        \\ \hline
%         \hline
%     \end{tabular}
% \end{center}
\section{Auswertung}
\label{sec:auswertung}

\section{Diskussion und Zusammenfassung}
\label{sec:diskussion_zusammenfassung}
% %Aufzählung was scheiße glaufen is
\begin{itemize}
    \item Mein Experiment hat ned funktioniert
    \item Irgendwas is da gewaltig daneben gangen
\end{itemize}

% Literaturtabelle
\bibliographystyle{unsrt}
\bibliography{Literatur} %Todo .bib befüllen zb.: mit JabRef (Empfehlung der Redaktion)

\listoffigures

\listoftables

\end{document}